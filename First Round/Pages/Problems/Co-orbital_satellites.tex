\section{Co-orbital satellites (50 Points)}

This question applies a method of determining the masses of two approximately co-orbital satellites developed by Dermott and Murray in 1981.

Suppose that two small satellites of masses $m_1$ and $m_2$ are approximately co-orbital (moving on very similar orbits) around a large central body of mass $M$, with $m_1,\, m_2\ll M$. At any instant, the orbits of the satellites may be approximated as circular Keplerian orbits with radii $r_1$ and $r_2$ respectively, although $r_1$ and $r_2$ will vary slightly over time due to the mutual gravitational interaction between the satellites.

Figure 4 depicts the shapes of the orbits in the rotating reference frame with zero angular momentum, centred on the central body. We denote by $\theta$ the angle $\measuredangle m_1Mm_2$, while $R$, $x_1$, and $x_2$ denote the mean orbital radius and radial deviations of the satellites.

Throughout this problem, write all answers in an inertial reference frame.

\emph{Hint:} $(1+x)^\alpha \approx 1 + \alpha x$ for $\alpha x \ll 1$

First we will determine the value of $\frac{m_1}{m_2}$.
\begin{enumerate}[label=(\alph*)]
    \item Write down the angular momentum $L_i$ of the satellite with mass $m_i$ when its circular orbit has radius $r_i$.\marginnote{\textbf{(3pt)}}
    \item The satiletes total angular momentum $L_1+L_2$ is conserved. Let $x_1$,\, $x_2$ $\ll R$ be the distances as shown in Figure 4. Find a simple relation between the ratios $\frac{m_1}{m_2}$ and $\frac{x_1}{x_2}$.\marginnote{\textbf{(8pt)}}\\

    \begin{solution}
    Esta es la solución de a
    \end{solution}

% \emph{Hint: You may use that $(1+x)^\alpha = 1+\alpha x$ for $|\alpha x|\ll1$.}\\\\
Now, we will try and determine the value of $m_1 + m_2$. For next parts, we will use the actual barycenter of the system, which may not be exactly at the center of the planet.
    \item The individual angular momenta of the satellites $m_1$ and $m_2$ will vary over time due to their gravitational interactions. Show that the rate of change of the angular momentum of the second satellite is given by
\[\dfrac{\Delta L_2}{\Delta t} \approx -\dfrac{Gm_1m_2}{R}h(\theta)\,\,\,\,\,\,\,\, \text{where} \,\,\,\,\,\,\,\, h(\theta)=\left[\dfrac{\cos{\left(\frac{\theta}{2}\right)}}{4\sin^2{\left(\frac{\theta}{2}\right)}}-\sin{\theta}\right] \marginnote{\textbf{(18pt)}}\]
    \item Show that $s = r_2 - r_1$ satisfies
\[\dfrac{\Delta s}{\Delta t} \approx -2\sqrt{\dfrac{G}{MR}}(m_1+m_2)h(\theta) \marginnote{\textbf{(8pt)}}\]
% You may use the following expression
% \[\dfrac{\Delta L_i}{\Delta t} = \dfrac{1}{2}m_i \sqrt{\dfrac{GM}{r_i}}\left(\dfrac{\Delta r_i}{\Delta t} \right) \]
\item For the angle $\theta = \measuredangle m_1Mm_2$ as indicated on Figure 4, find an expression for \(\dfrac{\Delta \theta}{\Delta t}\)\marginnote{\textbf{(5pt)}}
%show that
%\[\dfrac{\Delta \theta}{\Delta t} \approx -\dfrac{3}{2} \sqrt{\dfrac{GM}{R^3}} \left(\dfrac{s}{R}\right) \marginnote{\textbf{(7pt)}}\]
\item Using the results above, find the relation between $\Delta s$ and $\Delta \theta$.\marginnote{\textbf{(2pt)}}

\item After integrating the expression above, we will obtain the result,
\[\overline{x}^2 \approx \dfrac{4 R^2}{3} \dfrac{m_1+m_2}{M}\left(\dfrac{1}{\sin\left( \frac{\theta_{\rm min}}{2}\right)}-2\cos \theta_{\rm min} -3\right)\]
where $\overline{x} =\dfrac{x_1+x_2}{2}$.

Epimetheus ($m_1$) and Janus ($m_2$) are two approximately co-orbital moons of Saturn. Detailed observations of their orbits have been performed by the Voyager 1 and Cassini spacecraft, which found that $R = \SI{150000}{\km}$, $x_1 = \SI{76}{\km}$ and $x_2 = \SI{21}{\km}$. The minimum distance between Janus and Epimetheus is \SI{13000}{\km}. The mass of Saturn is known to be \SI{5.7e26}{\kg}. Estimate the masses of Epimetheus and Janus. \marginnote{\textbf{(6pt)}}
% \textbf{Note:} It is possible to show that for
% \begin{align*}
% \overline{x} & =\dfrac{x_1+x_2}{2}\\ 
% \overline{x}^2 & \approx \dfrac{4R^2}{3}\dfrac{m_1+m_2}{M}\left(\left(\sin{\left(\dfrac{\theta_{min}}{2}\right)}\right)^{-1}-2\cos{(\theta_{min})}-3\right)
% \end{align*}    
% is satisfied. 
\end{enumerate}
 \begin{figure}[h]
 \centering
 \includegraphics[scale=0.9]{Photos/Q12-figure.pdf}
 \caption{This figure schematically depicts the shapes of the orbits in the rotating reference frame, selected such that in this frame the total angular momentum of the two satellites is zero.}
 \end{figure}

\newpage
 