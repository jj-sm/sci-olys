\section{Relativistic Beaming (50 Points)}

Consider an isotropic light source of frequency $f_R$ in a frame which is fixed to the source (i.e. rest frame). In this rest frame, consider a light ray emitted from the source that makes an angle $\theta_R$ with the $X$-axis. The light source is moving along positive $X$ direction with (relativistic) speed $v$ as measured in the lab frame. 
\begin{enumerate} [label=(\alph*)]
    \item  Find an expression for the frequency $f_L$ of this ray in the lab frame, and the cosine of the angle that this ray makes with the $X$-axis in the lab frame. \marginnote{\textbf{(11pt)}}
\\
\emph{Hint:} In relativistic mechanics, energy $E$ and momentum $p$ of a particle between rest and lab frame are related in the following way:
\begin{align*}
& \frac{E_L}{c}=\gamma\left(\dfrac{E_R}{c}+p_{x_R}\dfrac{v}{c}\right)\\
    & p_{x_L}=\gamma\left(p_{x_R}+\dfrac{E_Rv}{c^2}\right)\\
    &  p_{y_L}=p_{y_R} \\
    & p_{z_L}=p_{z_R}
\end{align*}
where:
\begin{equation*}
    \gamma=\dfrac{1}{\sqrt{1-\frac{v^2}{c^2}}}
\end{equation*}
    \item For the following cases: 
    \begin{enumerate}[label=\roman*)]
        \item $\theta_R=\ang{0}$ 
        \item $\theta_R=\cos^{-1}(-v/c)$ 
        \item $\theta_R=\ang{90}$ 
        \item $\theta_L=\ang{180}$
    \end{enumerate}
    draw direction vectors of the beam in $XY$ plane of the rest frame as well as separately in $X'Y'$ plane of the lab frame. \marginnote{\textbf{(4 pt)}}
\end{enumerate}
    In accretion disks around black holes, the charged particles are orbiting at relativistic speeds and in their rest frames may be considered as isotropic point sources of light. Consider such a particle $K$ in a circular orbit of radius $r$ and angular speed $\omega$ around a central object located at $O$ (see figure).
\begin{center}
\includegraphics[scale=0.4]{Photos/Screenshot 2022-06-18 at 9.27.24 PM.png}
\end{center}
Let us assume that our lab frame is fixed to an observer located at a point $P$ on the $OY$ axis, which is stationary with respect to $O$. $OP=R\gg r$. Let $t_{L0}= t_{R0} = 0$ correspond to the moment when K is seen crossing the OX axis. As $K$ is moving with relativistic speed, the duration $\Delta t_R$ measured by an observer in the rest frame of the source $K$ is related to the duration measured in the lab frame $\Delta t_L$ at $P$ by the expression $\Delta t_L=\gamma \Delta t_R$.

\begin{enumerate} [label=(c)]
    \item Derive an expression for $f_L$ as a function of $t_L$ ($t_L>R/c$)? \marginnote{\textbf{(7pt)}}
\end{enumerate}
Let us consider a fraction of the light from the source that is emitted in an infinitesimal solid angle $\Delta\Omega_R=- \Delta(\cos \theta_R) \cdot \Delta\phi$ in the direction making an angle $\theta_R$ with respect to the $X$ axis in the rest frame, as it is shown on the figure below.
 \begin{center}
 \includegraphics[scale=0.4]{Photos/Screenshot 2022-07-04 at 8.50.44 PM.png}
 \end{center}
 \begin{enumerate} [label=(d)]
    \item  Show that, as measured in the lab frame
%  It can be shown that the corresponding solid angle in a lab frame (in which rest frame in moving with speed $v$ along $X$ axis) is:
\[\Delta\Omega_L=\dfrac{\Delta\Omega_R}{\gamma^2\left(1+\dfrac{v}{c}\cos\theta_R\right)^2}\marginnote{\textbf{(10pt)}}\]
\end{enumerate}

 \begin{enumerate} [label=(e)]
    \item If the intrinsic luminosity of the light source is $L$, what is the energy flux $F_L$ observed by the observer at point $P$ at the moment $t_L$ $(t_L>R/c)$? \marginnote{\textbf{(15pt)}}\\
\textbf{Hint:} In the rest frame of the source, you may assume $N_R$ number of photons are directed within the solid angle $\Delta \Omega_R$ during the time interval $\Delta t_R$.
% \\textcolor{blue}{Probably this question can be split in two first to calculate $N (t_L)$ then intensity}
%  \paragraph{}
%  \textbf{Hint:} The small change of $cos(\theta)$ due to small change in $\theta\rightarrow\theta+\Delta \theta$:
% \[\Delta(cos(\theta))=cos(\theta+\Delta\theta)-cos(\theta)\approx-sin(\theta)\Delta\theta\]
\end{enumerate}
\begin{enumerate}[label=(f)]
\item Charged particles in the relativistic beam shot from the supermassive black hole at the centre of the galaxy M87 have speeds up to $0.95c$. What would be the maximum and minimum amplification factor for the energy flux for a relativistic beam from M87? \marginnote{\textbf{(3pt)}}

\end{enumerate}