\thispagestyle{empty}


\begin{minipage}{0.4\textwidth}
    \flushleft
    \vspace{-2cm}\includegraphics[height=1.6cm]{Photos/Logo OCAAA-8.jpeg}
\end{minipage}
% Logo on the right
\begin{minipage}{0.5\textwidth}
    \flushright
    \vspace{-2cm}
    \includegraphics[height=1cm]{Photos/logoUAN.png}
\end{minipage}

% Centered and boxed text
\vspace{-1.8cm}
\begin{center}
    \boxed{\textbf{PT-SEL-\the\year00}}
\end{center}
\vspace{0.5cm}

\rule{\linewidth}{0.5mm}

\begin{centering}
    \vspace{0.3cm}
    \textbf{\large{Datos del Participante}}\\
    \vspace{0.4cm}
\end{centering}
% Name, Grade, etc.
\noindent
\textbf{APELLIDOS Y NOMBRES} \underline{\hspace{12.5cm}}\hfill \\
\vspace{0.1cm}

\textbf{GRADO} \underline{\hspace{4cm}}\hfill 
\textbf{COLEGIO} \underline{\hspace{9.21cm}}\hfill \\
\vspace{0.1cm}

\textbf{EMAIL} \underline{\hspace{8cm}}\hfill 
\textbf{TELÉFONO} \underline{\hspace{5cm}}\hfill \\
\vspace{0.1cm}

\textbf{CIUDAD} \underline{\hspace{4.5cm}}\hfill 
\textbf{DEPT.} \underline{\hspace{4.5cm}}\hfill 
\textbf{EDAD:} \underline{\hspace{2cm}}\hfill


% Table for grading
\vspace{0.2cm}

\renewcommand{\arraystretch}{2}  % This makes the rows taller
\begin{center}
\begin{tabular}{|m{1cm}|m{1cm}|m{1cm}|m{1cm}|m{1cm}|}
    \hline
    \multicolumn{1}{|c|}{\textbf{1}} & \multicolumn{1}{c|}{\textbf{2}} & \multicolumn{1}{c|}{\textbf{3}} & \multicolumn{1}{c|}{\textbf{4}} & \multicolumn{1}{c|}{\textbf{5}} \\
    \hline
    & & & & \\
    \hline
\end{tabular}
    \vspace{-1.97cm}
    \begin{flushright}
    \begin{tabular}{|m{1cm}|}
        \hline
        \multicolumn{1}{|c|}{\textbf{Total}} \\
        \hline
        \\
        \hline
    \end{tabular}
    \end{flushright}
\end{center}

% Note
\begin{center}
\vspace{0.3cm}
\textit{Cuadro para calificación (no llenar)}
\end{center}

\rule{\linewidth}{0.5mm}

\begin{centering}
    \vspace{0.2cm}
    \textbf{\large{Instrucciones}}\\
\end{centering}

\begin{enumerate}
    \item No abra los enunciados hasta que se le indique.
    \item Llene los datos del participante en los espacios asignados.
    \item Se dispone de \textbf{\boxed{90}} minutos para el desarrollo de la prueba.
    \item Es válido el uso de calculadoras (No se permite el uso de calculadoras graficadoras o calculadoras programables).
    \item Es permitido el uso de instrumentos de escritura y hojas en blanco, las cuales se le proporcionarán.
    \item Escriba sus respuestas y operaciones de manera clara y ordenada en las hojas que se le proporcionarán, a menos que el enunciado indique lo contrario.
    \item Está prohibido el uso de apuntes o ayudas. En caso de dudas en cuanto a los enunciados de la prueba, el profesor de la OCAAA, podrá resolverlas.
    \item Recuerde que está permitido el uso de la tabla de constantes (TC-ES-2024-01).
    \item El jurado se reserva el derecho de no considerar soluciónes confusas o ilegibles. Así como el derecho de rechazar soluciones que generen dudas de honestidad en su aplicación.
\end{enumerate}

\rule{\linewidth}{0.5mm}

\begin{centering}
    \vspace{0.3cm}
    \textbf{\large{Código de Honor}}\\
    \vspace{0.2cm}
\end{centering}
Con mi firma certifico que este es mi trabajo personal y no recibí ayuda o colaboración ajena en el desarrollo de la prueba.

\vspace{0.1cm}
{\centering Firma del participante: \underline{\hspace{8cm}} \hfill \par}

\vspace{0.1cm}
\rule{\linewidth}{0.2mm}
\begin{centering}
    \vspace{0.4cm}
    \copyright\ \the\year \ - Olimpiada Colombiana de Astronomía, Astrofísica y Astronáutica. $|$ \LaTeX

\end{centering}